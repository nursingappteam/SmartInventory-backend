%Explain, at a high level, how you will implement a solution to the problem. Include a diagram of major components to the system (not a full architectural design, but a high level overview of the major system components and how a user or external system might interface). Avoid specific implementation details (operating system, programming languages, etc.). This section should occupy at least 1 full page.

The application will be a full stack application where the user interacts with an inventory interface that allows them to view and request items from the equipment inventory. The Application will allow for the admin, in this case the Simulation Inventory Specialist, to customize there own categories as needed as well as add items. These items will have a unique identifier or bar code as well as the location of said item. Some items that are worth upwards of \$5,000.00 are classified as assets and are tracked using UTA's own tracking system. Our database will conform to these codes and allow for customization of bar codes at entry. Possibly the most important part of this inventory system will be an interface that allows a Actor level user to edit the database. This is because the admin will have zero coding experience.

The flow of command will begin with an instructor requesting an item to be checked out. This will then be approved by the admin and marked in the GUI. The database will be updated to the changes. After the item is finished it will either be returned to the storage area or removed from the database depending on the item's classification.

Another key component to the inventory system will be a timer attached to all checked out items.  This is per the nursing departments rule of a 72 hour check out limit for assets and non-assets. The extension of this timer will also have to be reset about approval from Simulation Inventory Specialist. Request to extend might be able to take place in application. 

On top of the check out system we will need a system to retain the locations of all items as well as the instructor that has them checked out if applicable. This will include the room numbers and section of room if applicable. The minutiae of this will have to be further fleshed out as they are currently moving facilities. Also in the item description there will  need to be a section that shows purchase history. Related to this there will also need to be an option to set up a maintenance schedule for some of the items that require it.

Consumables will be another item that will have to be recorded. They will be a special category that will build up as items are ordered so adding to a number of current items in the system will be needed. After Items are checked out they will be decremented from the count. This should be automated to allow for ease of use on the part of the administrator.

The final known requirement will be to create mass checkouts or 'labs' that will allow for the admin to add multiple items in one cart and apply the same status of location and owner to all of them. Also in this will the adding the consumables that are automatically taken out of the running count.